\title{ALPSチュートリアル -- ALPSのインストール}

\begin{document}

\begin{frame}
  \titlepage
\end{frame}

\section*{Outline}
\begin{frame}
  \tableofcontents
\end{frame}

\section{ALPSの利用方法}

\subsection*{\redb\whiteb\greenb}
\begin{frame}[fragile]
  \frametitle{ALPSの利用方法}
  \begin{itemize}
    \setlength{\itemsep}{1em}
  \item すでにインストールされているものを利用
    \begin{itemize}
    \item MateriApps LIVE!
    \item 物性研システムB (sekirei)
    \end{itemize}
  \item バイナリパッケージの利用
    \begin{itemize}
    \item Debian Package (Debian Wheezy, Jessie) [\href{https://github.com/cmsi/MateriAppsLive/wiki/UsingMateriAppsInDebian}{設定方法}]
    \item MacPorts (コンパイル方法に不備があり動作が遅い)
    \end{itemize}
  \item MateriApps Installerスクリプトを利用
    \begin{itemize}
    \item \url{https://github.com/wistaria/MateriAppsInstaller}
    \item macOS、Linux PC (GCC, Intel)、富士通FX10、京コンピュータ等
    \end{itemize}
  \item 自分でインストール
  \end{itemize}
\end{frame}

\subsection*{\redb\whiteb\greenb}
\begin{frame}
 \frametitle{MateriApps LIVE!}
 \begin{itemize}
   \item \url{http://cmsi.github.io/MateriAppsLive/}
   \item 多数の計算科学アプリがインストール済のLive Debian/GNU Linuxシステム
   \item ALPSもすでにインストール済
   \item MateriApps LIVE! VirtualBox版: VirtualBox仮想化アプリにMateriApps LIVE!仮想ハードディスクイメージ(OVA)をインポートし, 仮想マシン上で起動
     \begin{itemize}
     \item 動作が少し遅く、メモリがより多く必要であるが、簡単・確実
     \item \alert{本チュートリアルではこちらを利用} [\href{https://www.slideshare.net/cms\_initiative/materiapps-live-65694832}{設定方法}]
     \end{itemize}
   \item MateriApps LIVE! ブート用USB: MateriApps LIVE! ブートイメージが格納されたUSBメモリから直接起動
     \begin{itemize}
     \item PCによっては起動がうまくいかないことも
     \end{itemize}
 \end{itemize}
\end{frame}

\section{ALPSのインストール}

\subsection*{\redm\whitem\greenb}
\begin{frame}
  \frametitle{ALPSの依存関係}
  \begin{itemize}
  \item 必須のもの\\
    \begin{tabular}{ll}
      CMake & \url{http://www.cmake.org/} \\
      Boost & \url{http://www.boost.org/} \\
      HDF5  & \url{http://www.hdfgroup.org/HDF5/} \\
    \end{tabular}
  \item 結果の解析に必要なもの \\
    \begin{tabular}{ll}
      BLAS/LAPACK & \url{http://www.netlib.org/} \\
      Python & \url{http://www.python.org/} \\
      Numpy & \url{http://www.numpy.org} \\
      Scipy & \url{http://www.scipy.org} \\
      Matplotlib & \url{http://matplotlib.org/}
    \end{tabular}
  \item あるとよいもの \\
    \begin{tabular}{ll}
      MPI & \url{http://www.mpi-forum.org/} \\
    \end{tabular}
  \end{itemize}
\end{frame}

\subsection*{\redm\whitem\greenb}
\begin{frame}[fragile,shrink]
  \frametitle{Debian系Linuxへのインストール例}
  \begin{enumerate}
  \item 必要なライブラリをapt-getでインストール(rootでの作業)
\begin{semiverbatim}
$ sudo apt-get install cmake-curses-gui libhdf5-dev \\
  liblapack-dev mpi-default-dev python-matplotlib \\
  python-scipy libboost-all-dev
\end{semiverbatim}
  \item ALPS webページからソースをダウンロードして展開
  \item ALPSのビルドとインストール
\begin{semiverbatim}
$ mkdir build && cd build
$ cmake -DCMAKE_INSTALL_PREFIX=${HOME}/alps \\
  $HOME/somewhere/alps...
$ make
$ ctest
$ make install
\end{semiverbatim}
  \end{enumerate}
\end{frame}

\subsection*{\redm\whitem\greenb}
\begin{frame}[fragile,shrink]
  \frametitle{MateriApps Installerを使ったmacOSでのインストール例}
  \begin{enumerate}
  \item Xcodeコマンドラインツールをインストール
\begin{semiverbatim}
$ sudo xcode-select —install
\end{semiverbatim}
  \item MateriApps Installerをダウンロード
\begin{semiverbatim}
$ wget https://github.com/wistaria/MateriAppsInstaller/\\
  archive/master.tar.gz
$ tar zxvf master.tar.gz
\end{semiverbatim}
  \item 
    必要なライブラリを\href{http://www.macports.org/}{MacPorts}でインストール(rootでの作業)
\begin{semiverbatim}
$ sudo sh MateriAppsInstaller-master/macosx/ports.sh
\end{semiverbatim}
  \item BoostとALPSのビルド
\begin{semiverbatim}
$ mkdir -p $HOME/materiapps $HOME/build
$ sh MateriAppsInstaller-master/00_env/default.sh
$ sh MateriAppsInstaller-master/25_boost/macos.sh
$ sh MateriAppsInstaller-master/25_boost/link.sh
$ sh MateriAppsInstaller-master/70_alps/macos.sh
$ sh MateriAppsInstaller-master/70_alps/link.sh
\end{semiverbatim}
  \end{enumerate}
\end{frame}

\end{document}
