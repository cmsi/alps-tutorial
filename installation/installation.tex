\title{ALPSチュートリアル -- ALPSのインストール}

\begin{document}

\begin{frame}
  \titlepage
\end{frame}

\section*{Outline}
\begin{frame}
  \tableofcontents
\end{frame}

\section{ALPSがインストールされているシステム}
\subsection*{\redm\whiteb\greenb}

\begin{frame}[fragile]
  \frametitle{ALPSがインストールされているシステム}
  \begin{itemize}
  \item 物性研システムB (sekirei)
  \item 京 (k)
  \item 東大FX10 (oakleaf-fx)
  \item 京大Cray XE6 (camphor)
  \item 九大X86クラスタ (tatara)
  \item ...
  \item 他にも順次インストール中 (アカウントをいただければ出張インストールします)
  \end{itemize}
\begin{semiverbatim}
参考: {\footnotesize \url{https://github.com/wistaria/installer/wiki}}
\end{semiverbatim}
\end{frame}

\section{ALPSのインストール}
\subsection*{\redm\whiteb\greenb}

\begin{frame}
  \frametitle{ALPSの依存関係}
  \begin{itemize}
  \item 必須のもの\\
    \begin{tabular}{ll}
      CMake & \url{http://www.cmake.org/} \\
      Boost & \url{http://www.boost.org/} \\
      HDF5  & \url{http://www.hdfgroup.org/HDF5/} \\
    \end{tabular}
  \item 結果の解析に必要なもの \\
    \begin{tabular}{ll}
      BLAS/LAPACK & \url{http://www.netlib.org/} \\
      Python & \url{http://www.python.org/} \\
      Numpy & \url{http://www.numpy.org} \\
      Scipy & \url{http://www.scipy.org} \\
      Matplotlib & \url{http://matplotlib.org/}
    \end{tabular}
  \item あるとよいもの \\
    \begin{tabular}{ll}
      MPI & \url{http://www.mpi-forum.org/} \\
    \end{tabular}
  \end{itemize}
\end{frame}

\begin{frame}[fragile,shrink=10]
  \frametitle{Debian系Linuxへのインストール}
  \begin{enumerate}
  \item 必要なライブラリをapt-getでインストール(rootでの作業)
\begin{semiverbatim}
$ sudo apt-get install cmake-curses-gui libhdf5-dev \\
  liblapack-dev mpi-default-dev python-matplotlib \\
  python-scipy libboost-all-dev
\end{semiverbatim}
  \item ALPSのビルドとインストール
\begin{semiverbatim}
\$ wget -O - http://exa.phys.s.u-tokyo.ac.jp/archive/MateriApps/apt/pool/non-free/a/alps/alps_20150402~r7566.orig.tar.gz | tar zxf -
\$ mkdir build && cd build
\$ cmake -DCMAKE_INSTALL_PREFIX=${HOME}/alps \\
  $HOME/alps-20150402~r7566
\$ make
\$ ctest
\$ make install
\end{semiverbatim}
  \end{enumerate}
\end{frame}

\begin{frame}[fragile,shrink=10]
  \frametitle{Mac OS X (Marvericks)でのインストール}
  \begin{enumerate}
  \item Xcodeコマンドラインツールをインストール(rootでの作業)
\begin{semiverbatim}
\$ sudo xcode-select —install
\end{semiverbatim}
  \item 必要なライブラリを\href{http://www.macports.org/}{MacPorts}でインストール(rootでの作業)
\begin{semiverbatim}
\$ sudo port install gcc48
\$ sudo port select --set gcc mp-gcc48
\$ sudo port install openmpi-gcc48
\$ sudo port select --set mpi openmpi-gcc48-fortran
\$ sudo port install python27 py27-ipython py27-scipy py27-matplotlib \\
  py27-zmq py27-jinja2 cmake hdf5-18 +threadsafe wget
\end{semiverbatim}
  \item ALPSのビルドとインストール
\begin{semiverbatim}
\$ wget -O - http://exa.phys.s.u-tokyo.ac.jp/archive/source/\\
  alps-20140623-r7482.tar.gz | tar zxf -
\$ wget -O - http://sourceforge.net/projects/boost/files/\\
  boost/1.54.0/boost_1_54_0.tar.bz2 | tar jxf -
\$ mkdir build && cd build
\$ cmake -DCMAKE_INSTALL_PREFIX=${HOME}/alps \\
  -DCMAKE_CXX_COMPILER=g++ -DCMAKE_C_COMPILER=gcc \\
  -DCMAKE_Fortran_COMPILER=gfortran \\
  -DBoost_ROOT_DIR=$HOME/boost_1_54_0 \\
  -DPYTHON_INTERPRETER=python2.7 $HOME/alps-20140623-r7482
\$ make
\$ ctest
\$ make install
\end{semiverbatim}
  \end{enumerate}
\end{frame}

%% \begin{frame}[fragile,shrink=10]
%%  \frametitle{Mac OS X用のCMakeの設定例 (alps-cmake.sh)}
%%  \begin{semiverbatim}
%% #!/bin/bash
%% PREFIX=${HOME}/opt/alps
%% ALPS_SRC=${HOME}/sources/archive/alps-20140623-r7482
%% BOOST_SRC=${HOME}/sources/archive/boost_1_54_0

%% cmake -DCMAKE_INSTALL_PREFIX=${PREFIX} \\
%%   -DCMAKE_BUILD_TYPE=Release \\
%%   -DALPS_ENABLE_OPENMP=ON -DALPS_ENABLE_OPENMP_WORKER=ON \\
%%   -DPYTHON_INTERPRETER=python2.7 \\
%%   ${ALPS_SRC}
%%  \end{semiverbatim}
%% \end{frame}

\section{MateriApps LIVE!}
\subsection*{\redm\whiteb\greenb}

\begin{frame}
 \frametitle{MateriApps LIVE!}
 \begin{itemize}
   \item \url{http://cmsi.github.io/MateriAppsLive/}
   \item USBメモリから直接ブートできる Linux システム (Debian Live Linux)
   \item ALPSもすでにインストール済
   \item MateriApps LIVE! USBからのブート
     \begin{itemize}
     \item 展示会・講習会などでモニター配布中
     \item \href{https://github.com/cmsi/MateriAppsLive/wiki/HowToMakeLiveUSB}{自分でUSBへ書き込み}
     \end{itemize}
   \item 仮想マシンでMateriApps LIVE!を利用
     \begin{itemize}
     \item ISOイメージをダウンロード、仮想マシン(\href{https://github.com/cmsi/MateriAppsLive/wiki/Using-MateriApps-LIVE!-by-VMWare}{VMWare}, \href{https://github.com/cmsi/MateriAppsLive/wiki/KnownProblems}{VirtualBox}等)をブート
     \item 仮想マシン用イメージ(VMWare, VirtualBox, AWS)準備中
     \end{itemize}
 \end{itemize}
\end{frame}

\end{document}
