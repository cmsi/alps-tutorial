\lstset{language={Python}}

\title{ALPSチュートリアル: Introduction to ALPS Python}

\begin{document}

\begin{frame}
 \titlepage
\end{frame}

\section*{Outline}
\begin{frame}
 \tableofcontents
\end{frame}

\section{ALPS python}
\begin{frame}[t,fragile]
 \frametitle{Launching Python}

 \alert{ALPS を Python のモジュールとして利用することができます.}

 その準備として ALPS Python へ PATH を通す必要があります.

 phi 上で ALPS Python へ PATH を通すには次のコマンドを実行してください.

\begin{lstlisting}
bash> source /opt/nano/alps/alpsvars-20121208-r6630.sh
\end{lstlisting}


 自分で ALPS をインストールした場合,まず ALPS へ PATH を通してください.
 そこに alpspython というコマンドがインストールされているので,それを実行
 することで ALPS Python へ PATH が通せます.

\begin{lstlisting}
bash> alpspython
\end{lstlisting}

\end{frame}

\section{Preparing the simulation}
\begin{frame}[t,fragile]
 \frametitle{Importing the ALPS modules}

\begin{itemize}
 \item ALPS Python を Python のインタープリタから利用してみましょう.
       できれば iPython を使ってください.
       
       \begin{lstlisting}
$ ipython
>>> import pyalps
       \end{lstlisting}
       なにもエラーが出なければこれで準備終了です.


 \item Tab 補完で pyalps のモジュールや関数などが列挙できます.

       \begin{lstlisting}
>>> al.[ここで Tab]
Display all 102 possibilities? (y or n)
al.CycleColors   
al.CycleMarkers  
...            
       \end{lstlisting}
\end{itemize}
 
\end{frame}


\begin{frame}[t,fragile]
\frametitle{Preparing the input}

1 回のシミュレーションに必要なパラメータセットを辞書型で与えます.


この例では 温度 T を 1.5, 2, 2.5 と変化させて 3 回のシミュレーションを行うためのパラメータセットを準備しています.parms は要素として 3 個の辞書型の要素を持つリストになります.

\begin{lstlisting}
>>> parms = []
>>> for t in [1.5,2,2.5]:
...    parms.append({'LATTICE': "square lattice", 
...    'T': t,  'J': 1 ,
...    'THERMALIZATION': 1000, 'SWEEPS': 100000,
...    'UPDATE': "cluster", 'MODEL': "Ising",
...    'L': 8})
>>>
\end{lstlisting}

\end{frame}

% ここはいらない?
% \begin{frame}[t,fragile]
% \frametitle{Preparing the input}
% \begin{lstlisting}
% >>> from parms import *
% >>> print parms
% [{'LATTICE': 'square lattice', 'T': ...
% \end{lstlisting}
% \begin{itemize}
% \item Wiki ではインタープリタ上でパラメータを入力していますが,ここでは import を使ってファイルから読み込みました.
% \item カレントディレクトリに parms.py というファイル名で先ほどの内容のスクリプトを作成して置きます.これを import します.
% \item phi 上では LATTICE\_LIBRARY, MODEL\_LIBRARY の指定が必要です
% \item 辞書型を要素に持つリストが定義されています
% \end{itemize}
% \end{frame}

\begin{frame}[t,fragile]
\frametitle{Preparing the input}
先程作ったパラメータファイルを XML 形式のインプットファイルに変換します
\begin{lstlisting}
>>> input_file = pyalps.writeInputFiles('parm1',parms)
>>> print input_file  # 入力ファイル名が返されいる
parm1.in.xml
>>> (ctrl-z)
[1]+ 停止 python2.7
[halm@phi ~]# ls
ALPS.xsl parm1.in.xml parm1.task1.in.xml parm1.task2.in.xml parm1.task3.in.xml

\end{lstlisting}
\begin{itemize}
\item python を一時停止してカレントディレクトリを確認すると入力ファイルがいくつかできています.
  \begin{itemize}
  \item parm1.in.xml はシミュレーション全体の入力ファイルです
  \item parm1.task\# .in.xml は各タスク毎の入力ファイルです
  \end{itemize}
\end{itemize}

\end{frame}

\section{Running the simulation}
\begin{frame}[t,fragile]
\frametitle{Running the simulation on a serial machine}
\begin{lstlisting}
>>> pyalps.runApplication('spinmc',
       input_file,Tmin=5,writexml=True)
spinmc parm1.in.xml --Tmin 5 --write-xml
Generic classical Monte Carlo program using local or cluster updates
...
Checkpointing Simulation 3
Finished with everything.
(0, 'parm1.out.xml')
\end{lstlisting}
\begin{itemize}
\item 実行が終了すると parm1.out.xml, parm1.task1.out.h5, $\cdots$ といったファイルが出力されています.
\end{itemize}
 
\end{frame}

\begin{frame}[t,fragile]
\frametitle{Running the simulation on a parallel machine}
\begin{lstlisting}
>>> pyalps.runApplication('spinmc',input_file,Tmin=5,writexml=True,MPI=4)
\end{lstlisting}
\begin{itemize}
\item MPI=4 で MPI プロセス 4 並列でシミュレーションを実行することを指示しています.
\end{itemize}

\end{frame}

\section{Loading the simulation results}
\begin{frame}[t,fragile]
\frametitle{Getting the result files}
\begin{lstlisting}
>>> result_files = pyalps.getResultFiles(prefix='parm1')
>>> print result_files
['./parm1.task2.out.xml', './parm1.task3.out.xml', './parm1.task1.out.xml']
\end{lstlisting}
\begin{itemize}
\item カレントディレクトリ内からプレフィックスが 'parm1' のファイルを正規表現によるパターンマッチで探してファイル名を出力します.
\end{itemize}

\end{frame}

\begin{frame}[t,fragile]
\frametitle{Loading the results}
指定したファイル名のリストから,指定した物理量(複数可)を取り出す
\begin{lstlisting}
>>> data = pyalps.loadMeasurements(result_files,['|Magnetization|','Magnetization^2'])
>>> data[0]  # data の中身をみてみます
[x=[0]
y=[0.083339021352 +/- 0.00038780838614]
props={'observable': '|Magnetization|', 'THERMALIZATION': 10000.0, 'J': -1.0, ...}, 
x=[0]
y=[0.00816759554762 +/- 7.41432666026e-05]
props={'observable': 'Magnetization^2', 'THERMALIZATION': 10000.0, 'J': -1.0, ...}]
\end{lstlisting}
\end{frame}

\begin{frame}[t,fragile]
\frametitle{データの構造}
シミュレーション結果を表すクラス
\begin{itemize}
\item \textbf{DataSet}: class 名
  \begin{itemize}
  \item x: plot 用のインデックス (最初はデフォルト値 0 が入っている)
  \item y: 物理量の平均値および標準誤差
  \item props: 物理量名およびパラメータなどが Python の辞書形式で入っている
  \end{itemize}
\end{itemize}

\begin{lstlisting}
# 物理量の平均値と標準誤差を取り出してみます
>>> data[0][0].y
array([0.083339021352 +/- 0.00038780838614], dtype=object)
>>> data[0][0].y.mean
0.08333902135203504
>>> data[0][0].y.error
0.00038780838614
\end{lstlisting}
\end{frame}

\section{Plotting the results}
\begin{frame}[t,fragile]
\frametitle{Plotting the results}
\begin{lstlisting}
>>> plotdata = pyalps.collectXY(data,'T','|Magnetization|')
>>> plotdata[0].x
array([ 1.5,  2. ,  2.5])
>>> plotdata[0].y
array([0.986621683313 +/- 5.59763992999e-05,
       0.911889051072 +/- 0.000288443276697,
       0.605402476752 +/- 0.0013356753695], dtype=object)
>>> plotdata[0].props
{'observable': '|Magnetization|', '...
\end{lstlisting}
\begin{itemize}
\item data から pyalps.collectXY により x 軸として 'T', y 軸として '$|$Magnetization$|$' を取り出しています.
\end{itemize}
\end{frame}

\begin{frame}[t,fragile]
\frametitle{Plotting in Python using matplotlib}
\begin{lstlisting}
>>> pyalps.plot.plot(plotdata) # データをプロット.
[<matplotlib.lines.Line2D at 0x109425dd0>]
>>> plt.xlim(0,3)             # x 軸の範囲を設定
>>> plt.ylim(0,1)             # y 軸の範囲を設定
>>> plt.title('Ising model')  # 図のタイトル
>>> plt.savefig('ising.pdf')  # ファイルへ出力
>>> plt.show()                # X11 などで画面に表示
\end{lstlisting}
\includegraphics[scale=0.2]{ising.pdf}
\end{frame}

\begin{frame}[t,fragile]
\frametitle{Converting to other formats}
ほかのデータフォーマットに変換できます
\begin{lstlisting}
>>> print pyalps.plot.convertToText(plotdata)
>>> print pyalps.plot.makeGnuplotPlot(plotdata)
>>> print pyalps.plot.makeGracePlot(plotdata)
\end{lstlisting}

text 形式の出力例
\begin{lstlisting}
# 
# X: T
# Y: |Magnetization|
1.5	0.986496240665 +/- 2.82049446481e-05
2.0	0.912126100521 +/- 0.000350201474667
2.5	0.603552651599 +/- 0.00146115425377
\end{lstlisting}

\end{frame}

\section{Evaluating data}
\begin{frame}[t,fragile]
\frametitle{Example of evaluating data}
binder 比 $<m^2>/<|m|>^2$ の計算をしてみます
\begin{lstlisting}
>>> binder = pyalps.DataSet()
>>> binder.props = pyalps.dict_intersect([d[0].props for d in data])
>>> binder.x = [d[0].props['T'] for d in data]
>>> binder.y = [d[1].y[0]/(d[0].y[0]*d[0].y[0]) for d in data]
>>> print binder
\end{lstlisting}
\begin{itemize}
\item 空のデータセットを作ります
\item 複数の辞書から key:val ともに一致する項目を抜き出す
\item データから x 軸となる 'T' の数列をとりだす.
\item $d[0]$ から $m^2$, $d[1]$ から $|m|$ を取り出して binder 比を計算
\end{itemize}
\end{frame}

\end{document}
