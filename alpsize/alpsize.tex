\title{ALPSチュートリアル -- アプリケーションのALPS化}

\begin{document}

\begin{frame}
  \titlepage
\end{frame}

% \begin{frame}
%   \begin{columns}[t]
%     \begin{column}{.3\textwidth}
%     \end{column}
%     \begin{column}{.7\textwidth}
%       \tableofcontents
%     \end{column}
%   \end{columns}
% \end{frame}

\begin{frame}
  \frametitle{ALPSize 全体の流れ}
  \begin{itemize}
  \item 01-04 C++でのプログラミング
  \item 05-07 ALPSライブラリ単体の利用
  \item \alert{08 ALPSスケジューラ}
  \end{itemize}
  \begin{block}{}
    ALPSスケジューラで単純な並列化と中断・再開
  \end{block}
\end{frame}

\begin{frame}
  \frametitle{CMake}
  \begin{itemize}
  \item クロスプラットフォームなビルドツール\\
    Windowsにも対応
  \item Makefileを生成
  \item 高速な並列ビルド
  \item 簡単なテスト
  \end{itemize}
\end{frame}

\begin{frame}[fragile]
  \frametitle{alpsizeの準備}
\begin{semiverbatim}
\$ mkdir -p alpsize/XX
\$ cd alpsize/XX
\end{semiverbatim}
  \begin{block}{}
    ソースディレクトリとビルド・テストディレクトリは分けましょう
  \end{block}
\end{frame}

\begin{frame}[fragile]
  \frametitle{00 CMake化 (1)}
  簡単なプログラムをCMakeを利用してビルドする
\begin{semiverbatim}
\$ cmake /opt/nano/alps/alpsize/00_cmake/
\$ make
\$ ctest
\$ ./hello
\end{semiverbatim}
\end{frame}

\begin{frame}[fragile,shrink=10]
  \frametitle{00 CMake化 (2)}
\begin{semiverbatim}
\$ cat /opt/nano/alps/alpsize/00_CMake/CMakeLists.txt
\alert<1>{cmake_minimum_required(VERSION 2.8 FATAL_ERROR)}
\alert<1>{project(hello NONE)}

# find ALPS Library
\alert<1>{find_package(ALPS REQUIRED PATHS ${ALPS_ROOT_DIR} $ENV{ALPS_HOME} NO_SYSTEM_ENVIRONMENT_PATH)}
\alert<1>{message(STATUS "Found ALPS: ${ALPS_ROOT_DIR} (revision: ${ALPS_VERSION})")}
\alert<1>{include(${ALPS_USE_FILE})}

# enable C and C++ compilers
\alert<2>{enable_language(CXX)}

# rule for generating 'hello world' program
\alert<3>{add_executable(hello hello.C)}
\alert<4>{add_alps_test(hello)}
\end{semiverbatim}
\end{frame}

\begin{frame}[fragile]
  \frametitle{00 CMake化 (3)}
add\_alps\_testは以下と同等
\begin{semiverbatim}
\$ ./hello /opt/nano/alps/alpsize/00_CMake/hello.ip > hello.tmp
\$ diff -u hello.tmp /opt/nano/alps/alpsize/00_CMake/hello.op
\$ rm hello.tmp
\end{semiverbatim}
実行バイナリを期待される出力と比較するテスト
\begin{semiverbatim}
\$ cat /opt/nano/alps/alpsize/00_CMake/hello.op
hello, world
\end{semiverbatim}

\end{frame}

\begin{frame}[fragile]
  \frametitle{01 Cプログラム}
  Wolffアルゴリズムのプログラム
\begin{semiverbatim}
\$ cmake /opt/nano/alps/alpsize/01_original-c/
\$ make
\$ ctest
\$ ./wolff
\end{semiverbatim}
\end{frame}

\begin{frame}[fragile]
  \frametitle{02 C++プログラム}
  WolffアルゴリズムをC++で
\begin{semiverbatim}
\$ cmake /opt/nano/alps/alpsize/02_basic-cpp/
\$ make
\$ ctest
\$ ./wolff
\end{semiverbatim}
\end{frame}

\begin{frame}[fragile]
  \frametitle{03 STLの利用}
  STLを利用する
\begin{semiverbatim}
\$ cmake /opt/nano/alps/alpsize/03_stl/
\$ make
\$ ctest
\$ ./wolff
\end{semiverbatim}
\end{frame}

\begin{frame}[fragile]
  \frametitle{04 Boostの利用 (1)}
  Boostライブラリを利用する
\begin{semiverbatim}
\$ cmake /opt/nano/alps/alpsize/04_boost/
\$ make
\$ ctest
\$ ./wolff
\end{semiverbatim}
\end{frame}

\begin{frame}[fragile,shrink=10]
  \frametitle{04 Boostの利用 (2)}
\begin{semiverbatim}
...
#include <boost/random.hpp>
...
// random number generator
boost::mt19937 eng(SEED);
boost::variate_generator<boost::mt19937&, boost::uniform_real<> >
random_01(eng, boost::uniform_real<>());
\end{semiverbatim}
\end{frame}

\begin{frame}[fragile]
  \frametitle{05 ALPS Parameters}
  ALPS Parameterを利用する
\begin{semiverbatim}
\$ cmake /opt/nano/alps/alpsize/05_parameters/
\$ make
\$ ctest
\$ ./wolff
\end{semiverbatim}
\end{frame}

\begin{frame}[fragile] \frametitle{Parameterライブラリ}
  プログラムに与えるパラメータを解釈するライブラリ
  \begin{columns}
    \begin{column}{6cm}
\begin{semiverbatim}
LATTICE = "chain lattice";
L = 16,

SEED = 2873
// C++ style comment
SWEEPS = 4096;
THERMALIZATION = \alert<1>{SWEEPS/8};
/* C style comment */
\{ T = 2; Sq = 2*\alert<2>{PI}/3; \}
\{ T = 1.8; \}
\end{semiverbatim}
    \end{column}
    \begin{column}{6cm}
      \begin{itemize}
      \item 四則演算可能
      \item $\pi$ を文字で指定
      \item 空行も正しく認識
      \item C風, C++風のコメント
      \item \{ \} で囲んだ変数は異なるセット
      \item 空白を適切に認識
      \item 改行, セミコロン, コンマで変数を区別
      \end{itemize}
    \end{column}
  \end{columns}
\end{frame}

\begin{frame}[fragile,shrink=5] \frametitle{Parameter ライブラリの利用法}
\begin{semiverbatim}
#include <boost/foreach.hpp>
#include <alps/parameter.h>
...
int main(int argc, char** argv) \{
...
  alps::ParameterList plist = read_param(argc, argv);
  BOOST_FOREACH(alps::Parameters& p, plist) \{
    \alert<1>{double a = p["a"];}
    \alert<2>{double b = p.value_or_default("b", 0.5);}
...
  \}
...
\}
\end{semiverbatim}
\end{frame}

\begin{frame}[fragile]
  \frametitle{06 ALPS Alea}
  ALPS Aleaを利用する
\begin{semiverbatim}
\$ cmake /opt/nano/alps/alpsize/06_alea/
\$ make
\$ ctest
\$ ./wolff
\end{semiverbatim}
\end{frame}

\begin{frame}[fragile,shrink=10] \frametitle{Alea ライブラリの利用法}
  マルコフ連鎖における平均値, 分散, 自己相関を計算するライブラリ
\begin{semiverbatim}
#include <alps/alea.h>
...
alps::ObservableSet obs;
...
obs << alps::RealObservable("Energy");
...
for (int i = 0; i != N; ++i) \{
  double e = energy(); // calclulate something
  obs.reset(i == THERM); // when thermalized
  obs["Energy"] << e;
\}
...
std::cout << obs["Energy"];
\end{semiverbatim}
\end{frame}

\begin{frame}[fragile,shrink=10] \frametitle{Tips: 出力を制御する}
\begin{semiverbatim}
// Gnuplot 風出力
std::cout << 1./param["beta"] << " "
          << obs["Energy"].mean() << " "
          << obs["Energy"].variance() << std::endl;
\end{semiverbatim}
\end{frame}

\begin{frame}[fragile]
  \frametitle{07 ALPS Lattice}
  ALPS Latticeを利用する
\begin{semiverbatim}
\$ cmake /opt/nano/alps/alpsize/07_lattice/
\$ make
\$ ctest
\$ ./wolff
\end{semiverbatim}
\end{frame}

\begin{frame}[fragile] \frametitle{Lattice ライブラリの利用法}
\begin{semiverbatim}
#include <alps/lattice.h>
...
class ising : public alps::graph_helper<> \{
 public:
  ising(alps::Parameters const& p)
   : \alert{graph_helper<>(p)} \{\}
...
\};
\end{semiverbatim}

\end{frame}

\begin{frame}[fragile,shrink=30] \frametitle{Lattice の定義}
\begin{semiverbatim}
<LATTICE name="square lattice" dimension="2">
  <PARAMETER name="a" default="1"/>
  <BASIS><VECTOR>a 0</VECTOR><VECTOR>0 a</VECTOR></BASIS>
  <RECIPROCALBASIS><VECTOR>2*pi/a 0</VECTOR><VECTOR>0 2*pi/a</VECTOR></RECIPROCALBASIS>
</LATTICE>
<UNITCELL name="simple2d" dimension="2">
  <VERTEX/>
  <EDGE><SOURCE vertex="1" offset="0 0"/><TARGET vertex="1" offset="0 1"/></EDGE>
  <EDGE><SOURCE vertex="1" offset="0 0"/><TARGET vertex="1" offset="1 0"/></EDGE>
</UNITCELL>
<LATTICEGRAPH name = "square lattice">
  <FINITELATTICE>
    <LATTICE ref="square lattice"/>
    <PARAMETER name="W" default="L"/>
    <EXTENT dimension="1" size="L"/>
    <EXTENT dimension="2" size="W"/>
    <BOUNDARY type="periodic"/>
  </FINITELATTICE>
  <UNITCELL ref="simple2d"/>
</LATTICEGRAPH>
\end{semiverbatim}
\end{frame}

\begin{frame}[fragile,shrink=10] \frametitle{Lattice ライブラリの便利な関数}
\begin{semiverbatim}
std::vector<double> d(\alert{num_sites()});
std::vector<double> J(\alert{num_bonds()});
...
double energy = 0.;
BOOST\_FOREACH(site_descriptor s, \alert{sites()}) \{
  BOOST\_FOREACH(bond_descriptor b, \alert{neighbor_bonds(s)}) \{
    energy += J[b] * d[\alert{source(b)}] * d[\alert{target(b)}];
  \}
\}
...
boost::mt19937 rng;
boost::uniform_int<> dst(0, \alert{num_bonds()}-1);
J[\alert{bond(dst(rng))}] = 0;
Jtemp = param["J" + \alert{bond_type(b)}];
\end{semiverbatim}
  \begin{itemize}
  \item sites(), bonds() は BOOST\_FOREACH と組み合わせる
  \item index(s) や index(b) で数字に変換
  \item bond(n) や site(n) で *\_descriptor に変換
  \end{itemize}
\end{frame}

\begin{frame}[fragile]
  \frametitle{08 ALPS Scheduler (1)}
  ALPS Schedulerを利用する
\begin{semiverbatim}
\$ cmake /opt/nano/alps/alpsize/08_scheduler/
\$ make
\$ ctest
\$ ./hello
\$ ./wolff
\end{semiverbatim}
\end{frame}
\begin{frame}[fragile,shrink=10]
  \frametitle{08 ALPS Scheduler (2)}
\begin{semiverbatim}
class wolff_worker : public alps::parapack::lattice_mc_worker<> \{
private:
  typedef alps::parapack::lattice_mc_worker<> super_type;

public:
  wolff_worker(alps::Parameters const& params) : super_type(params);
  virtual ~wolff_worker();

  void init_observables(alps::Parameters const&, alps::ObservableSet& obs);
  bool is_thermalized() const;
  double progress() const;

  \alert{void run(alps::ObservableSet& obs);}
  void save(alps::ODump& dp) const;
  void load(alps::IDump& dp);
\};
\end{semiverbatim}
\end{frame}

\begin{frame}[fragile]
  \frametitle{08 ALPS Scheduler (3)}
\begin{semiverbatim}
class wolff_evaluator : public alps::parapack::simple_evaluator \{
public:
  wolff_evaluator(alps::Parameters const&);
  void evaluate(alps::ObservableSet& obs) const;
\};
\end{semiverbatim}
\end{frame}

\begin{frame}[fragile]
  \frametitle{08 ALPS Scheduler (4)}
\begin{semiverbatim}
#include <alps/parapack/parapack.h>

int main(int argc, char** argv) \{
  return alps::parapack::start(argc, argv);
\}
\end{semiverbatim}
\end{frame}

\begin{frame}[fragile]
  \frametitle{ALPS Fortran}
  ALPS SchedulerをFortranプログラムに
\end{frame}

\end{document}

%%% Local Variables: 
%%% mode: japanese-latex
%%% TeX-master: nil
%%% coding: utf-8
%%% End: 
