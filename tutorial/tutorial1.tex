%-*- coding:utf-8 -*-

\title{ALPSチュートリアル -- アプリケーション実習(1)}

\begin{document}

\lstset{language={C++},showspaces=false,rulecolor=\color[cmyk]{0, 0.29,0.84,0}}

\begin{frame}
  \titlepage
\end{frame}

\section*{Outline}
\begin{frame}[t,fragile]
   \tableofcontents
\end{frame}

\section{シミュレーションの前に}

\subsection*{\redb\whiteb\greenm}
\begin{frame}[t,fragile]{シミュレーションの前に -- 共通の操作}
  \begin{itemize}
    % \setlength{\itemsep}{1em}
  \item ターミナルを開く
    \begin{itemize}
    \item \alert{MateriAps LIVE!では、スタートメニュー ⇒「Accessories」⇒「LXTerminal」}
    \item macOSでは「ターミナル」アプリケーションを開く
    \item リモートのワークステーションを使う場合には、手元でターミナルを開き、SSHログインする
    \end{itemize}

  \item パスの設定
    \begin{itemize}
    \item \alert{MateriAps LIVE!ではパスは設定済み}
    \item その他の環境では alpsvars.sh を実行する必要がある。例)
\begin{lstlisting}
$ source /usr/local/alps/bin/alpsvars.sh
\end{lstlisting}
    \end{itemize}
    
  \item チュートリアルファイルのコピー
    \begin{itemize}
    \item \alert{MateriApps LIVE!では、チュートリアルファイルは /usr/share/alps/tutorials/ にある}
    \item その他の環境では、ALPSのインストール先に tutorials ディレクトリがあるはず
    \end{itemize}
  \end{itemize}
\end{frame}

\section{二次元強磁性イジング模型}

\subsection*{\redb\whiteb\greenm}
\begin{frame}[t,fragile]{二次元強磁性イジング模型の相転移シミュレーション(1)}
  \begin{itemize}
    % \setlength{\itemsep}{1em}
  \item 相転移を起こす磁性体の最も単純化した模型
    \[
      {\cal H} = - J \sum_{\langle i,j \rangle} \sigma_i \sigma_j \qquad (\sigma = \pm 1)
      \]
    \item 臨界温度( $T_{\rm c} = 2/\log(1+\sqrt{2}) = 2.269\cdots$)で連続相転移が起こる
    \item 高温 ($T > T_{\rm c}$): 常磁性相、エントロピーの効果で長距離秩序なし
    \item 低温 ($T < T_{\rm c}$): 強磁性相、相互作用の効果で長距離秩序が現れる
    \item 臨界点で比熱や帯磁率は発散 (臨界現象)
    \item simplemc を用いて、古典モンテカルロ法(メトロポリス法)による平衡状態のシミュレーションをやってみる
  \end{itemize}
\end{frame}

\subsection*{\redb\whiteb\greenm}
\begin{frame}[t,fragile]{二次元強磁性イジング模型の相転移シミュレーション(2)}
  \begin{itemize}
    % \setlength{\itemsep}{1em}
  \item チュートリアルファイルのコピー
\begin{lstlisting}
$ cp -r /usr/share/alps/tutorials $HOME
$ cd $HOME/tutorials/mc-09-snapshot
\end{lstlisting}
  \item パラメータの変換 (一群のXMLファイル parm9a.*.in.xml の生成)
\begin{lstlisting}
$ parameter2xml parm9a
\end{lstlisting}
  \item シミュレーションの実行
\begin{lstlisting}
$ simplemc parm9a.in.xml
\end{lstlisting}
\item 結果のプロット
\begin{lstlisting}
$ python plot9a.py
\end{lstlisting}
\item プロットウインドウ(3枚)を閉じる
  \end{itemize}
\end{frame}

\subsection*{\redb\whiteb\greenm}
\begin{frame}[t,fragile]{パラメータファイルの中身 (parm9a)}
  \begin{itemize}
    \setlength{\itemsep}{1em}
  \item ``less parm9a'' で中身を見てみる
    \begin{itemize}
    \item \verb+LATTICE="square lattice"+ : 正方格子を指定
    \item \verb+J=1+ : 強磁性相互作用 ($J>0$)
    \item \verb+ALGORITHM="ising"+ : イジング模型のシミュレータを使う
    \item \verb+SWEEPS=65536+ : モンテカルロステップ数
    \item \verb+L=8+ : 格子の一辺の長さ ($L=8, 16, 24$)
    \item \verb+{ T = 5.0 }+ : シミュレーションする温度のリスト
    \end{itemize}
  \item 中括弧に入っていないのは共通パラメータ
  \item 中括弧ごとにシミュレーションが実行される(温度毎に XML ファイル(*.task*.*)が作られる)
  \end{itemize}
\end{frame}

\subsection*{\redb\whiteb\greenm}
\begin{frame}[t,fragile,shrink]{プロットファイルの中身 (plot9a.py)}
  \begin{itemize}
    % \setlength{\itemsep}{1em}
  \item データファイルからの結果の抽出
\begin{lstlisting}
data = pyalps.loadMeasurements(pyalps.getResultFiles(prefix='parm9a'),
  ['Specific Heat','Magnetization Density^2','Binder Ratio of Magnetization'])
\end{lstlisting}
  \item システムサイズごとに整理
\begin{lstlisting}
for item in pyalps.flatten(data):
    item.props['L'] = int(item.props['L'])
magnetization2 = pyalps.collectXY(data, x='T',y='Magnetization Density^2’,
  foreach=['L'])
magnetization2.sort(key=lambda item: item.props['L'])
\end{lstlisting}
  \item matplotlib をつかってプロット
\begin{lstlisting}
pyplot.figure()
alpsplot.plot(magnetization2)
\end{lstlisting}
  \end{itemize}
\end{frame}

\subsection*{\redb\whiteb\greenm}
\begin{frame}[t,fragile]{イジング模型のスナップショット(スピン状態)の可視化}
  \begin{itemize}
    % \setlength{\itemsep}{1em}
  \item 常磁性相、臨界点近傍、強磁性相でのスピン状態を実際に見てみる
  \item パラメータファイル(parm9b)に、パラメタ SNAPSHOT\_INTERVAL が追加されている
  \item パラメータの変換
\begin{lstlisting}
$ parameter2xml parm9b
\end{lstlisting}
  \item シミュレーションの実行
\begin{lstlisting}
$ simplemc parm9b.in.xml
\end{lstlisting}
\item スナップショットの変換
\begin{lstlisting}
$ snap2vtk parm9b.*.snap
\end{lstlisting}
  \end{itemize}
\end{frame}

\subsection*{\redb\whiteb\greenm}
\begin{frame}[t,fragile]{ParaViewによるスナップショットの可視化(1)}
  \begin{itemize}
    % \setlength{\itemsep}{1em}
  \item ParaView を起動 (スタートメニュー⇒「Education」⇒「ParaView Viewer」
  \item file ⇒ open ⇒ parm9b.task1.clone1.16384.vtk を選択 ⇒ OK ⇒ Apply
    \begin{itemize}
    \item filters ⇒ common ⇒ glyph (あるいは地球儀のような形のアイコンをクリック)
    \item Glyph Type = Box, X Length = 0.08, Y Length = 0.08, Maximum Number of Points = 20000 ⇒ Apply
    \end{itemize}
  \item OpenGL window の右上の水平分割アイコンをクリック
  \item file ⇒ open ⇒ parm9b.task2.clone1.16384.vtk を選択 ⇒ OK ⇒ Apply
    \begin{itemize}
    \item filters ⇒ common ⇒ glyph (あるいは地球儀のような形のアイコンをクリック)
    \item Glyph を同様に設定
    \end{itemize}
  \end{itemize}
\end{frame}

\subsection*{\redb\whiteb\greenm}
\begin{frame}[t,fragile]{ParaViewによるスナップショットの可視化(2)}
  \begin{itemize}
    % \setlength{\itemsep}{1em}
  \item OpenGL window の右上の水平分割アイコンをクリック
  \item file ⇒ open ⇒ parm9b.task3.clone1.16384.vtk を選択 ⇒ OK ⇒ Apply
    \begin{itemize}
      \item filters ⇒ common ⇒ glyph (あるいは地球儀のような形のアイコンをクリック)
      \item Glyph を同様に設定
    \end{itemize}
  \item カメラのリンク
    \begin{itemize}
    \item ウインドウのサイズが同じになるように調整
    \item 真ん中のウインドウをクリック ⇒ Tools ⇒ Add Camera Link ⇒ 左のウインドウをクリック
    \item (あるいは 真ん中のウインドウ右クリック ⇒ Link Camera... を選択 ⇒ 左のウインドウをクリック)
    \item 同様に右のウインドウも左のウィンドウとリンク
    \end{itemize}
  \end{itemize}
\end{frame}

\section{二次元古典XY模型のスピン渦の可視化}

\subsection*{\redb\whiteb\greenm}
\begin{frame}[t,fragile]{二次元古典XY模型のスピン渦の可視化}
  \begin{itemize}
    % \setlength{\itemsep}{1em}
  \item パラメータファイルの変換・シミュレーションの実行・スナップショットの変換
\begin{lstlisting}
$ parameter2xml parm9c
$ simplemc parm9c.in.xml
$ snap2vtk parm9c.*.snap
\end{lstlisting}
パラメータファイル中で\verb+ALGORITHM="xy"+となっていることに注意
  \item ParaView による可視化
    \begin{itemize}
      \item file ⇒ open ⇒ parm9c.task1.clone1.16384.vtk を選択 ⇒ OK
      \item properties タブ ⇒ Apply
      \item Glyph を追加
      \item Glyph Type = Arrow, Tip Radius = 0.2, Shaft Radius = 0.06, Translate = -0.1 0 0, Scale = 0.2 0.2 0.2 ⇒ Apply
    \end{itemize}
  \end{itemize}
\end{frame}

\section{二次元三角格子反強磁性古典ハイゼンベルグ模型}

\subsection*{\redb\whiteb\greenm}
\begin{frame}[t,fragile]{二次元三角格子反強磁性古典ハイゼンベルグ模型}
  \begin{itemize}
    % \setlength{\itemsep}{1em}
  \item パラメータファイルの変換・シミュレーションの実行・スナップショットの変換
\begin{lstlisting}
$ parameter2xml parm9d
$ simplemc parm9d.in.xml
$ snap2vtk parm9d.*.snap
\end{lstlisting}
パラメータ\verb+LATTICE+, \verb+J+, \verb+ALGORITHM+, \verb+H+に注目
  \item ParaView による可視化
    \begin{itemize}
      \item file ⇒ open ⇒ parm9d.task1.clone1.16384.vtkを選択 ⇒  OK
      \item properties タブ ⇒ Apply
      \item Glyph を追加し Arrow を設定
      \item Displayタグ ⇒ Color by ⇒ Glyph Vector ⇒ Z ⇒ Properties タグ
      \item もう一枚 (parm9d.task2.clone1.16384.vtk) も表示、カメラをリンク
    \end{itemize}
  \end{itemize}
\end{frame}

\end{document}
