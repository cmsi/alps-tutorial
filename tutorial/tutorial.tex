\title{ALPSアプリケーション実行チュートリアル}

\begin{document}

\begin{frame}
  \titlepage
\end{frame}

% \begin{frame}
%   \begin{columns}[t]
%     \begin{column}{.3\textwidth}
%     \end{column}
%     \begin{column}{.7\textwidth}
%       \tableofcontents
%     \end{column}
%   \end{columns}
% \end{frame}

\begin{frame}[fragile,shrink=10]
  \frametitle{準備}
  \begin{enumerate}
  \item<1-> phiへログイン
\begin{semiverbatim}
\$ ssh -X guestXX@phi.aics.riken.jp
\end{semiverbatim}
  \item<2-> 本日利用するチュートリアルをコピー
\begin{semiverbatim}
[guestXX@phi ~]\$ cp -r /opt/nano/alps/tutorials/mc-02-susceptibilities .
[guestXX@phi ~]\$ cp -r /opt/nano/alps/tutorials/mc-03-magnetization .
\end{semiverbatim}
  \item<3-> 環境変数の設定
\begin{semiverbatim}
[guestXX@phi ~]\$ echo "source /opt/nano/alps/alpsvars.sh" >> ~/.bashrc
\end{semiverbatim}
  \item<4-> phiXXへログイン
\begin{semiverbatim}
[guestXX@phi ~]\$ ssh -X phiXX
\end{semiverbatim}
  \end{enumerate}
  \begin{alertblock}{ALPS wikiのチュートリアル}
    \url{http://alps.comp-phys.org/mediawiki/index.php/ALPS_2_Tutorials:Overview/ja}
  \end{alertblock}
\end{frame}

\begin{frame}[fragile]
  \frametitle{ALPSの実行シナリオ}
  \begin{enumerate}
  \item コマンドライン(旧来からの方法)
\begin{semiverbatim}
\$ parameter2xml \textit{param}
\$ loop \textit{param}.in.xml
\end{semiverbatim}
  \item \alert{Python} (新しい方法)
    \begin{itemize}
    \item パラメータの準備からグラフ作成まで統一的に
    \item 対話的にもコマンドでも実行可能
    \end{itemize}
  \item VisTrails
  \end{enumerate}
  \begin{columns}
    \begin{column}{.7\textwidth}
      \begin{center}
        \begin{alertblock}{}
          Python(Python)で対話的に実習を行います
        \end{alertblock}
      \end{center}
    \end{column}
  \end{columns}
\end{frame}

\begin{frame}[fragile]
  \frametitle{ALPSチュートリアル: MC-02 (1)}
  \begin{enumerate}
  \item<1-> Xの転送ができるか確認
\begin{semiverbatim}
 \$ gnome-about
\end{semiverbatim}
  \item<1-> Pythonの実行
\begin{semiverbatim}
 \$ cd mc-02-susceptibilities
 \$ ipython
\end{semiverbatim}
  \end{enumerate}
\end{frame}

\begin{frame}[fragile,shrink=5]
  \frametitle{ALPSチュートリアル: MC-02 (2)}
  \begin{enumerate}
  \item<2-> ALPS共通の準備
\begin{semiverbatim}
In [1]: import pyalps
\end{semiverbatim}
  \item<3-> グラフを生成する準備
\begin{semiverbatim}
In [2]: import matplotlib.pyplot as plt
In [3]: import pyalps.plot
...
In [10]: plt.show()
\end{semiverbatim}
  \item<3-> Xの転送ができない場合
\begin{semiverbatim}
In [2]: import matplotlib
In [3]: matplotlib.use('PDF')
In [4]: import matplotlib.pyplot as plt
In [5]: import pyalps.plot
...
In [13]: plt.savefig('out.pdf')
\end{semiverbatim}
  \end{enumerate}
\end{frame}

\begin{frame}
  \frametitle{ディレクトリ構成}
  \begin{tabular}{ll}
    /opt/nano/alps/alpsvars.sh & ALPS設定スクリプト \\
    /opt/nano/alps/tutorials & アプリケーションのチュートリアル \\
    /opt/nano/alps/alpsize   & ALPS化のチュートリアル
  \end{tabular}
  \begin{alertblock}{}
    alpsvars.shはALPSがインストールされたスパコン全てにあります \\
    参考: google ``List of ALPS Preinstalled Supercomputers''
\end{alertblock}
\end{frame}

\end{document}

%%% Local Variables: 
%%% mode: japanese-latex
%%% TeX-master: nil
%%% coding: utf-8
%%% End: 
