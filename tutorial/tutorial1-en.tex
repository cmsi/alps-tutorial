%-*- coding:utf-8 -*-

\title{ALPS Tutorial \\ 03 -- Application Tutorial (1)}

\begin{document}

\lstset{language={C++},showspaces=false,rulecolor=\color[cmyk]{0, 0.29,0.84,0}}

\begin{frame}
  \titlepage
\end{frame}

\section*{Outline}
\begin{frame}[t,fragile]
   \tableofcontents
\end{frame}

\section{Before starting simulations}

\subsection*{\redb\whiteb\greenm}
\begin{frame}[t,fragile]{Before starting simulations -- common operations}
  \begin{itemize}
    % \setlength{\itemsep}{1em}
  \item Open terminal
    \begin{itemize}
    \item \alert{On MateriAps LIVE!: Start Menu $\Rightarrow$ ``System Tools'' $\Rightarrow$ ``LXTerminal''}
    \item On macOS: Open ``Terminal'' application
    \item If using a remote workstation, open the terminal and SSH to the remote node
    \end{itemize}

  \item PATH setting
    \begin{itemize}
    \item \alert{On MateriAps LIVE!: PATH has been configured already}
    \item In other environment, you have to run alpsvars.sh. ex)
\begin{lstlisting}
$ source /usr/local/alps/bin/alpsvars.sh
\end{lstlisting}
    \end{itemize}
    
  \item Tutorial files
    \begin{itemize}
    \item \alert{On MateriApps LIVE!: Tutorial files can be found in /usr/share/alps/tutorials/}
    \item In other environment, there should be ``tutorials'' directory in the place where ALPS is installed
    \end{itemize}
  \end{itemize}
\end{frame}

\section{Two-dimensional ferromagnetic Ising model}

\subsection*{\redb\whiteb\greenm}
\begin{frame}[t,fragile]{Phase transition in two-dimensional ferromagnetic Ising model (1)}
  \begin{itemize}
    % \setlength{\itemsep}{1em}
  \item Most simplest model that exhibits a phase transiton
    \[
      {\cal H} = - J \sum_{\langle i,j \rangle} \sigma_i \sigma_j \qquad (\sigma = \pm 1)
      \]
    \item Second-order phase transition happens  at critical temperature ($T_{\rm c} = 2/\log(1+\sqrt{2}) = 2.269\cdots$)
    \item At high temperatures ($T > T_{\rm c}$): paramagnetic phase, no long-range order due to entropy effect
    \item At low temperatures ($T < T_{\rm c}$): ferromagnetic phase, long-range order due to interaction between spins
    \item Specific heat and susceptibility diverge at the critical point (critical phenomena)
    \item Simulation of equilibrium state by simplemc classical Monte Carlo code (Metropolis method)
  \end{itemize}
\end{frame}

\subsection*{\redb\whiteb\greenm}
\begin{frame}[t,fragile]{Phase transition in two-dimensional ferromagnetic Ising model (2)}
  \begin{itemize}
    % \setlength{\itemsep}{1em}
  \item Copy tutorial file
\begin{lstlisting}
$ cp -r /usr/share/alps/tutorials/mc-09-snapshot .
$ cd mc-09-snapshot
\end{lstlisting}
  \item Convert parameter file into XML files parm9a.*.in.xml
\begin{lstlisting}
$ parameter2xml parm9a
\end{lstlisting}
  \item Execute simulation
\begin{lstlisting}
$ simplemc parm9a.in.xml
\end{lstlisting}
\item Plot results
\begin{lstlisting}
$ python plot9a.py
\end{lstlisting}
\item Close three plot windows
  \end{itemize}
\end{frame}

\subsection*{\redb\whiteb\greenm}
\begin{frame}[t,fragile]{Paramter file (parm9a)}
  \begin{itemize}
    \setlength{\itemsep}{1em}
  \item Look at the contents by ``less parm9a''
    \begin{itemize}
    \item \verb+LATTICE="square lattice"+ : specify square lattice
    \item \verb+J=1+ : ferromagnetic interaction ($J>0$)
    \item \verb+ALGORITHM="ising"+ : use Ising model simulator
    \item \verb+SWEEPS=65536+ : Monte Carlo steps
    \item \verb+L=8+ : linear extent of lattice ($L=8, 16, 24$)
    \item \verb+{ T = 5.0 }+ : list of temperatures
    \end{itemize}
  \item Variables not enclosed in \{ \} are common parameters
  \item Simulation is performed for each \{ \} (XML file (*.task*.*) is created for each temperature)
  \end{itemize}
\end{frame}

\subsection*{\redb\whiteb\greenm}
\begin{frame}[t,fragile,shrink]{Python plot script (plot9a.py)}
  \begin{itemize}
    % \setlength{\itemsep}{1em}
  \item Extraction results from data files
\begin{lstlisting}
data = pyalps.loadMeasurements(pyalps.getResultFiles(prefix='parm9a'),
  ['Specific Heat','Magnetization Density^2','Binder Ratio of Magnetization'])
\end{lstlisting}
  \item Organize by system sizes
\begin{lstlisting}
for item in pyalps.flatten(data):
    item.props['L'] = int(item.props['L'])
magnetization2 = pyalps.collectXY(data, x='T',y='Magnetization Density^2',
  foreach=['L'])
magnetization2.sort(key=lambda item: item.props['L'])
\end{lstlisting}
  \item Plot using matplotlib
\begin{lstlisting}
pyplot.figure()
alpsplot.plot(magnetization2)
\end{lstlisting}
  \end{itemize}
\end{frame}

\subsection*{\redb\whiteb\greenm}
\begin{frame}[t,fragile]{Visualization of snapshot of Ising spins (1)}
  \begin{itemize}
    % \setlength{\itemsep}{1em}
  \item Look at spin state in paramagnetic phase, in critical point, and in ferromagnetic phase
  \item Parameter SNAPSHOT\_INTERVAL is added in the parameter file (parm9b)
  \item Convert parameter file
\begin{lstlisting}
$ parameter2xml parm9b
\end{lstlisting}
  \item Execute simulation
\begin{lstlisting}
$ simplemc parm9b.in.xml
\end{lstlisting}
\item Convert snapshot
\begin{lstlisting}
$ snap2vtk parm9b.*.snap
\end{lstlisting}
  \end{itemize}
\end{frame}

\subsection*{\redb\whiteb\greenm}
\begin{frame}[t,fragile]{Visualization of snapshot of Ising spins (2)}
  \begin{itemize}
    % \setlength{\itemsep}{1em}
  \item Start ParaView (Start Menu $\Rightarrow$ ``Education'' $\Rightarrow$ ``ParaView Viewer''
  \item file $\Rightarrow$ open $\Rightarrow$ choose parm9b.task1.clone1.16384.vtk $\Rightarrow$ OK $\Rightarrow$ Apply
    \begin{itemize}
    \item filters $\Rightarrow$ common $\Rightarrow$ glyph (or click globe-like icon)
    \item Glyph Type = Box, X Length = 0.08, Y Length = 0.08, Maximum Number of Points = 20000 $\Rightarrow$ Apply
    \end{itemize}
  \item Click horizontal division icon at upper right corner
  \item file $\Rightarrow$ open $\Rightarrow$ choose parm9b.task2.clone1.16384.vtk $\Rightarrow$ OK $\Rightarrow$ Apply
    \begin{itemize}
    \item filters $\Rightarrow$ common $\Rightarrow$ glyph (or click globe-like icon)
    \item Set Glyph in the same way
    \end{itemize}
  \end{itemize}
\end{frame}

\subsection*{\redb\whiteb\greenm}
\begin{frame}[t,fragile]{Visualization of snapshot of Ising spins (3)}
  \begin{itemize}
    % \setlength{\itemsep}{1em}
  \item Click horizontal division icon at upper right corner
  \item file $\Rightarrow$ open $\Rightarrow$ choose parm9b.task3.clone1.16384.vtk $\Rightarrow$ OK $\Rightarrow$ Apply
    \begin{itemize}
      \item filters $\Rightarrow$ common $\Rightarrow$ glyph (or click globe-like icon)
      \item Set Glyph in the same way
    \end{itemize}
  \item Link cameras
    \begin{itemize}
    \item Adjust windows to become the same size
    \item Click middle window $\Rightarrow$ Tools $\Rightarrow$ Add Camera Link $\Rightarrow$ click left window
    \item Link right window to left one similarly
    \end{itemize}
  \end{itemize}
\end{frame}

\section{Visualization of spin vortices}

\subsection*{\redb\whiteb\greenm}
\begin{frame}[t,fragile]{Visualization of spin vortex in two-dimensional classical XY model}
  \begin{itemize}
    % \setlength{\itemsep}{1em}
  \item Convert parameter file, execute simulation, convert snapshots
\begin{lstlisting}
$ parameter2xml parm9c
$ simplemc parm9c.in.xml
$ snap2vtk parm9c.*.snap
\end{lstlisting}
Note that \verb+ALGORITHM="xy"+ in the parameter file
  \item Visualization using ParaView
    \begin{itemize}
      \item file $\Rightarrow$ open $\Rightarrow$ choose parm9c.task1.clone1.16384.vtk $\Rightarrow$ OK
      \item properties tab $\Rightarrow$ Apply
      \item Add Glyph
      \item Glyph Type = Arrow, Tip Radius = 0.2, Shaft Radius = 0.06, Translate = -0.1 0 0, Scale = 0.2 0.2 0.2 $\Rightarrow$ Apply
    \end{itemize}
  \end{itemize}
\end{frame}

\section{Antiferromagnetic classical Heisenberg model on triangular lattice}

\subsection*{\redb\whiteb\greenm}
\begin{frame}[t,fragile]{Antiferromagnetic classical Heisenberg model on triangular lattice}
  \begin{itemize}
    % \setlength{\itemsep}{1em}
  \item Convert parameter file, execute simulation, convert snapshots
\begin{lstlisting}
$ parameter2xml parm9d
$ simplemc parm9d.in.xml
$ snap2vtk parm9d.*.snap
\end{lstlisting}
Note that \verb+LATTICE+, \verb+J+, \verb+ALGORITHM+, \verb+H+ in the parameter file
  \item Visualization using ParaView
    \begin{itemize}
      \item file $\Rightarrow$ open $\Rightarrow$ choose parm9d.task1.clone1.16384.vtk $\Rightarrow$  OK
      \item properties tag $\Rightarrow$ Apply
      \item Add Glyph and set Arrow
      \item Display tag $\Rightarrow$ Color by $\Rightarrow$ Glyph Vector $\Rightarrow$ Z $\Rightarrow$ Properties tag
      \item Display one more snapshot (parm9d.task2.clone1.16384.vtk) and link cameras
    \end{itemize}
  \end{itemize}
\end{frame}

\end{document}
